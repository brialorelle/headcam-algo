% Template for Cogsci submission with R Markdown

% Stuff changed from original Markdown PLOS Template
\documentclass[10pt, letterpaper]{article}

\usepackage{cogsci}
\usepackage{pslatex}
\usepackage{float}
\usepackage{caption}

% amsmath package, useful for mathematical formulas
\usepackage{amsmath}

% amssymb package, useful for mathematical symbols
\usepackage{amssymb}

% hyperref package, useful for hyperlinks
\usepackage{hyperref}

% graphicx package, useful for including eps and pdf graphics
% include graphics with the command \includegraphics
\usepackage{graphicx}

% Sweave(-like)
\usepackage{fancyvrb}
\DefineVerbatimEnvironment{Sinput}{Verbatim}{fontshape=sl}
\DefineVerbatimEnvironment{Soutput}{Verbatim}{}
\DefineVerbatimEnvironment{Scode}{Verbatim}{fontshape=sl}
\newenvironment{Schunk}{}{}
\DefineVerbatimEnvironment{Code}{Verbatim}{}
\DefineVerbatimEnvironment{CodeInput}{Verbatim}{fontshape=sl}
\DefineVerbatimEnvironment{CodeOutput}{Verbatim}{}
\newenvironment{CodeChunk}{}{}

% cite package, to clean up citations in the main text. Do not remove.
\usepackage{apacite}

% KM added 1/4/18 to allow control of blind submission


\usepackage{color}

% Use doublespacing - comment out for single spacing
%\usepackage{setspace}
%\doublespacing


% % Text layout
% \topmargin 0.0cm
% \oddsidemargin 0.5cm
% \evensidemargin 0.5cm
% \textwidth 16cm
% \textheight 21cm

\title{A longitudinal investigation of the social information in natural infant
visual experience}


\author{{\large \bf Bria Long (bria@stanford.edu)}  \AND {\large \bf George Kachergis (kachergis@stanford.edu)}  \AND {\large \bf Ketan Jay Agarwal (agrawalk@stanford.edu)}  \AND {\large \bf Michael C. Frank (mcfrank@stanford.edu)} \\  Department of Psychology, Street Address \\ Stanford, CA 91305 USA}

\begin{document}

\maketitle

\begin{abstract}
The faces and hands of infants' caregivers and other social partners
offer a rich source of social and causal information that may be
critical for infants' cognitive and linguistic development. Previous
work using manual annotation strategies and cross-sectional data has
found systematic changes in the proportion of faces and hands in the
egocentric perspective of young infants. The present research aims to
test the generality of these findings using the SAYcam dataset
(Sullivan, Mei, Perfors, Wojcik, \& Frank, n.d.), a longitudinal
collection of over 1700 headcam videos collected from three children
along a span of 6 to 32 months of age. To do so, we validate the use of
a modern convolutional neural network for pose detection (OpenPose) for
the detection of people, faces, and hands to analyze these naturalistic
infant egocentric videos. We then apply this model to the entire
dataset, analyzing the prevalence of faces across age, individuals, and
activity contexts. Overall, we find a higher prevalence of hands seen by
infants than previously reported, considerably variability in the
proportion of faces/hands seen across different locations (e.g., living
room vs.~kitchen), yet surprising consistency across both individual
children.

\textbf{Keywords:}
social cognition; face perception; infancy; head cameras; deep learning
\end{abstract}

\hypertarget{introduction}{%
\section{Introduction}\label{introduction}}

Infants are confronted by a blooming, buzzing onslaught of stimuli
(James, 1891) which they must learn to parse to make sense of the world
around them. Yet infants do not embark on this learning process alone:
infants are engaged in learning from their caregivers from early
infancy. From as early as 3 months of age, young infants follow overt
gaze shifts (Gredeback, Fikke, \& Melinder, 2010), and even newborns
prefer to look at faces with direct vs.~averted gaze (Farroni, Csibra,
Simion, \& Johnson, 2002), despite their limited acuity {[}CITE{]}.\\
Faces are thus likely to be an important conduit of social information
that scaffolds infants cognitive development. Given this importance,
developmentalists have long hypothesized {[}CITE{]} that faces are
prevalent in the visual experience of young infants. However, as even
the viewpoint of a walking child is not easily predicted by our own
adult intuitions (Clerkin, Hart, Rehg, Yu, \& Smith, 2017; Franchak,
Kretch, Soska, \& Adolph (2011); Yoshida \& Smith (2008)), researchers
have begun to record the egocentric views collected from infants and
toddlers wearing head-mounted cameras to test theories about the infant
perspective.

A growing body of work now demonstrates that the viewpoints of very
young infants--less than 4 months of age--are indeed dominated by
frequent, persistent views of the faces of their caregivers (Jayaraman,
Fausey, \& Smith, 2015; Jayaraman \& Smith, 2018; Sugden, Mohamed-Ali,
\& Moulson, n.d.). However, as infants mature, their perspective starts
to capture views of hands paired with the objects they are acting on
(Fausey, Jayaraman, \& Smith, 2016). As infants learn to use their own
hands to act on the world, they may focus on manual actions taken by
their social partners. Furthermore, caregivers may start to use their
hands more with communicative intent, directing infants attention with
pointing and gestures to particular events or objects during play (Yu \&
Smith, 2013).

The present research aims to test the generality of these findings using
the SAYcam dataset (Sullivan et al., n.d.), a longitudinal collection of
over 1700 headcam videos collected from three children along a span of 6
to 32 months of age. In addition to its size and longitudinal nature,
this dataset builds on those used in previous research in two key ways.
First, recordings were from a variety of many different naturalistic
contexts, encompassing infants' viewpoints during both activities
outside and inside the home. Second, the cameras used in this
longitudinal study encompassed a much wider field of view than those
typically used, allowing a more complete picture of the infant
perspective.

However, with hundreds of hours of footage (\textgreater{}30M frames),
this large dataset truly necessitates a shift to an automated annotation
strategy. Indeed, annotation of the frames extracted from egocentric
videos has been to be prohibitively time-consuming, meaning that many of
the frames are not inspected. For example, Fausey et al. (2016),
collected a total of 143 hours of head-mounted camera footage (15.5
million frames), of which one frame every five seconds was hand
annotated (by four coders), totalling 103,383 frames (per coder)--an
impressive number of annotations but nonetheless only 0.67\% of the
collected footage. To address this challenge, we first validate the use
of a modern computer vision model (Cao, Hidalgo, Simon, Wei, \& Sheikh,
2018) to automatically detect the presence of hands and faces from the
infant egocentric viewpoint. In particular, we focus on OpenPose (Cao et
al., 2018), a model optimized for jointly detecting human face, body,
hand, and foot keypoints (135 in total) that operates well on scenes
including multiple people even if they are partially-occluded (see
Figure 1).\\
We then apply these methods at scale to the larger dataset, allowing us
to analyze the proportion of faces and hands observed by each child
across age and activity context.

In the following paper, we first describe the dataset and the pose model
used in the following analyses and validate the use of this model for
extracting our key descriptive variables by comparing to a
human-annotated gold set of 24,000 frames. Next, we analyze key
descriptive variables, including those that have been previously
reported to vary across age, including the relative proportions of faces
vs.~hands and the sizes of the faces, over the entire dataset (30M
frames), finding a greater prevalence of hands than has been previously
reported.\\
We then investigate sources of variability in our more naturalistic
dataset that may explain these differences, including a diversity of
activity contexts as well as a larger field of view captured by our
cameras.

\hypertarget{method}{%
\section{Method}\label{method}}

\hypertarget{dataset}{%
\subsection{Dataset}\label{dataset}}

Videos captured by the headcam were 640x480 pixels, and a fisheye lens
was attached to the camera to increase 109 degrees horizontal x 70
degrees vertical. Children wore headcams at least twice weekly, for
approximately one hour per recording session. One weekly session was on
the same day each week at a roughly constant time of day, while the
other(s) were chosen arbitrarily at the participating family's
discretion. At the time of the recording, all three children were in
single-child households. Videos\footnote{All videos are available at
  \url{https://nyu.databrary.org/volume/564}} with technical errors or
that were not taken from the egocentric perspective were excluded from
the dataset. We analyze 1,636 videos, with a total duration of XX hours
(XX million frames).

\hypertarget{part-1-how-well-can-we-capture-social-information-using-computer-vision}{%
\subsection{Part 1: How well can we capture social information using
computer
vision?}\label{part-1-how-well-can-we-capture-social-information-using-computer-vision}}

\hypertarget{computer-vision-model}{%
\subsubsection{Computer vision model}\label{computer-vision-model}}

To automatically annotate the millions of frames in SAYcam, we use
OpenPose (Cao et al., 2018; Simon, Joo, Matthews, \& Sheikh, 2017), a
computer vision model optimized for jointly detecting human face, body,
hand, and foot keypoints (135 in total) that operates well on scenes
including multiple people even if they are partially-occluded. This
convolutional neural network (CNN)-based pose detector\footnote{\url{https://github.com/CMU-Perceptual-Computing-Lab/openpose}}
provided the locations of 18 body parts (ears, nose, wrists, etc.). The
system uses a CNN for initial anatomical detection and subsequently
applies part affinity fields (PAFs) for part association, producing a
series of body part candidates. The candidates are then matched to a
single individual and finally assembled into a pose; here, we only made
use of outputs of the face and hand detection modules.

\hypertarget{manual-annotation-strategy}{%
\subsubsection{Manual annotation
strategy}\label{manual-annotation-strategy}}

To test the validity of OpenPose's hand and face detections, we compared
the accuracy of these detections relative to human annotations of 24,000
frames selected uniformly at random from the 1,636 videos of two
children (S and A). Frames were jointly annotated for the presence of
faces and hands. These randomly sampled frames covered XX of the videos
present in the dataset.\\
A second set of coders recruited via AMT (Amazon Mechanical Turk)
additionally annotated XX frames; agreement with the primary coder was
XX\%.

\hypertarget{detection-accuracy-for-faces-and-hands}{%
\subsubsection{Detection accuracy for faces and
hands}\label{detection-accuracy-for-faces-and-hands}}

Overall, we found that face detections were slightly more accurate than
hand detections,

Precision and recall (F-score) variation across child/age for faces

Describe possible sources of variation that decrease scores for: -Faces:
weird viewpoints, occluded/side viewpoint, faces in books -Hands:
children's own hands, hands in books, side viewpoints

Describe additional child vs.~hand annotation; P/R/F variation across
child vs.~adult hands (better for adult hands, still OK for child hands)

\hypertarget{part-2-access-to-social-information-across-age}{%
\subsection{Part 2: Access to social information across
age}\label{part-2-access-to-social-information-across-age}}

Next, we analyzed the social information in view across the entire
dataset, looking specifically at the proportions of faces and hands that
were in view for each child. Data from videos were binned according to
the age of the child (in weeks) (see Figure XX). First, we saw that the
proportion of faces in view showed a moderate decrease across this age
range, both when analyzing the random 24K frames as well as the entire
dataset. While these trends appear somewhat different than those
observed in Fausey et al. (2016), note that Fausey et al. (2016)
included data from very young infants (starting at 2 months while),
while here the youngest videos coming from S and A around 6 and 9 months
of age, respectively. Similarly, our age range extends 8 months later
than those infants in Fausey et al. (2016), throughout a portion of
third year of life.

However, the most striking result from the dataset is a much greater
proportion of hands in view than have previously been reported. We found
this to be true across all ages, in both children, and regardless of
whether we analyzed human annotations (on the 24K random subset) or the
entire dataset.\\
One reason this could be the case is the much larger field of view that
was captured by the cameras used in this study: unlike previous studies,
our cameras were outfitted with a fish-eye lens in an attempt to capture
as much of the children's field of view as possible.\\
Thus, the field of view (FOV) of the fisheye lens used in Sullivan et
al. (n.d.) was much wider (109 degrees horizontal x 70 degrees vertical)
than the FOV of the lens used in Fausey et al. (2016) (69 x 41 degrees).
This larger field of view may have allowed the SAYcam cameras to capture
not only the presence of a social partner's hands interacting with
objects, but also the children's own hands, leading to more frequent
hand detections.

To assess this possibility, we obtained annotations for a subsample of
XX frames in which a hand was detected in the random gold set;
participants (recruited via AMT) were asked to draw bounding boxes
around children's hands and adult hands. Overall, we found that 34\% of
the hands detected by OpenPose in the random 24K sample were of
children's own hands (compared to 8\% reported in Fausey et al. (2016))
suggesting that this difference in field of view did contribute
substantially to the higher rate of detections in the frames. Heatmaps
of the bounding boxes obtained from these annotations can be seen in
Figure XX, showing that children's hands tended to appear in the lower
half of the frames.

We thus re-analyzed the entire dataset while restricting our analysis to
a smaller, middle portion of the frame comparable to the field of view
used in Fausey et al. (2016) (XX vs XX degrees). To do so, we excluded
hand detections that occurred in the bottom 40\% of the frame, while
retaining all detections that occurred in the top of the frame.\\
Overall, we found X, suggesting Y.

Intriguingly, we observed an unexpected trend in which the relative
proportion of faces vs.~hands increased during the third year of life;
in particular, this seemed\ldots{}(save for discussion?)

\hypertarget{variability-by-location}{%
\subsubsection{Variability by Location}\label{variability-by-location}}

Next we examine variation in the presence of hands and faces across
different locations. Of the XX videos, the content of 1,829 have been
manually annotated for filming location, activites taking place, and
visible objects (see Sullivan et al. (n.d.)). XX of the videos were
filmed in a single location, representing XX\% of the dataset and
roughly XX million frames (see Sullivan et al. (n.d.)). To give a sense
of the contexts the children experienced, the most frequent filming
locations were the living room (339 videos), bedroom (182), kitchen
(150), outside on property (129), child's bedroom (81), deck/porch (73),
hallway (70), and off property (57). Filming only took place twice in
the dining room (see Figure XX). Activities varied predictability by
these locations: for example, eating and XX occured in the kitchen,
whereas playtime was the dominant activity in the living room (see
Figure XX). Bria: analyze activity

Not this, but point out that eating was not frequent: The most frequent
activities were sitting (410), playing (375), being held (352), and
standing (297). Eating was the 11th most-frequent activity (117 videos).

(goldset, full dataset)

\hypertarget{discussion}{%
\section{Discussion}\label{discussion}}

We demonstrate the feasibility of using modern computer vision models to
vastly increase the efficiency of processing egocentric headcam footage,
allowing us to annotate 100\% of even very large datasets for the
presence and size of people, hands, and faces. We validate the use of
this model by comparing to a human-annotated gold set, and find X.

Analyzing this dataset has yielded a better understanding of infants'
evolving access and attention to social information.

Need to think about the child's viewpoint relative to actual FOV as well
as attention

Fausey 2016: 103,383 images; Here: 30,000,000 frames; 300 fold increase
in data

\hypertarget{acknowledgements}{%
\section{Acknowledgements}\label{acknowledgements}}

We would like to thank X and Y for helpful comments, and\ldots{}

\hypertarget{references}{%
\section{References}\label{references}}

\setlength{\parindent}{-0.1in} 
\setlength{\leftskip}{0.125in}

\noindent

\hypertarget{refs}{}
\leavevmode\hypertarget{ref-Cao2018openpose}{}%
Cao, Z., Hidalgo, G., Simon, T., Wei, S.-E., \& Sheikh, Y. (2018).
OpenPose: Realtime multi-person 2D pose estimation using Part Affinity
Fields. In \emph{ArXiv preprint arXiv:1812.08008}.

\leavevmode\hypertarget{ref-Farroni2002}{}%
Farroni, T., Csibra, G., Simion, F., \& Johnson, M. H. (2002). Eye
contact detection in humans from birth. \emph{Proceedings of the
National Academy of Sciences}, \emph{99}(14), 9602--9605.

\leavevmode\hypertarget{ref-Fausey2016}{}%
Fausey, C. M., Jayaraman, S., \& Smith, L. B. (2016). From faces to
hands: Changing visual input in the first two years. \emph{Cognition},
\emph{152}, 101--107.

\leavevmode\hypertarget{ref-Franchak2011}{}%
Franchak, J. M., Kretch, K. S., Soska, K. C., \& Adolph, K. E. (2011).
Head-mounted eye- tracking: A new method to describe infant looking.
\emph{Child Development}, \emph{82}(6), 1738--1750.

\leavevmode\hypertarget{ref-Gredeback2010}{}%
Gredeback, G., Fikke, L., \& Melinder, A. (2010). The development of
joint visual attention: A longitudinal study of gaze following during
interactions with mothers and strangers. \emph{Developmental Science},
\emph{13}(6), 839--848.

\leavevmode\hypertarget{ref-Jayaraman2015}{}%
Jayaraman, S., Fausey, C. M., \& Smith, L. B. (2015). The faces in
infant-perspective scenes change over the first year of life. \emph{PLoS
One}. \url{http://doi.org/10.1371/journal.pone.0123780}

\leavevmode\hypertarget{ref-Jayaraman2018}{}%
Jayaraman, S., \& Smith, L. B. (2018). Faces in early visual
environments are persistent not just frequent. \emph{Vision Research}.

\leavevmode\hypertarget{ref-Simon2017hand}{}%
Simon, T., Joo, H., Matthews, I., \& Sheikh, Y. (2017). Hand keypoint
detection in single images using multiview bootstrapping. In
\emph{CVPR}.

\leavevmode\hypertarget{ref-Sugden2014}{}%
Sugden, N. A., Mohamed-Ali, M. I., \& Moulson, M. C. (n.d.). I spy with
my little eye: Typical, daily exposure to faces documented from a
first-person infant perspective. \emph{Developmental Psychobiology},
\emph{56}(2), 249--261.

\leavevmode\hypertarget{ref-SAYcam}{}%
Sullivan, J., Mei, M., Perfors, A., Wojcik, E., \& Frank, M. (n.d.).
Head cameras on children aged 6 months through 31 months.

\leavevmode\hypertarget{ref-Yoshida2008}{}%
Yoshida, H., \& Smith, L. (2008). What's in view for toddlers? Using a
head camera to study visual experience. \emph{Infancy}, \emph{13}(3),
229--248.

\bibliographystyle{apacite}


\end{document}
